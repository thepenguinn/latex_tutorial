\documentclass{beamer}

\usepackage{minted}
\usepackage[size=custom,width=16,height=9,scale=0.5]{beamerposter}
\usepackage{tcolorbox}

% \usepackage{beamercolorthemedracula}

\usetheme{metropolis}
\usecolortheme{dracula}

\title{An Introduction to \LaTeX}
\date{\today}
\author{Daniel V Mathew}

\institute{RIT, Kottayam}

\newcommand\alertNote[1] {
    \begin{tcolorbox}[
            coltitle = draculafg,
            colbacktitle = violet,
            colback = violet,
            colframe = violet,
            title=Note,
            fonttitle=\bfseries,
            detach title,
        ]
        \begin{minipage}[t]{0.18\textwidth}
            \begin{flushleft}
                \tcbtitle
            \end{flushleft}
        \end{minipage}
        \begin{minipage}[t]{0.8\textwidth}
            {\color{draculafg} #1}
        \end{minipage}
    \end{tcolorbox}
    }

\begin{document}
\maketitle

\usemintedstyle{catppuccin-mocha}

\section{A Typical \LaTeX{} Document}

\begin{frame}[fragile]{Document Structure}

    \alert{latex} or \alert{tex} source files ends with \alert{.tex} extension.

    \begin{minted}[breaklines, autogobble, linenos] {latex}
        \documentclass[a4paper, 11pt]{article}
        % preamble <- a comment by the way.

        \begin{document}
        % document

        \end{document}
    \end{minted}

\end{frame}

\begin{frame}[fragile]{Comments}
    Comments in \LaTeX{} starts with a percent (\alert{\%}) sign.

    \begin{minted}[breaklines, autogobble, linenos] {latex}
        % Hai, this is a comment in LaTeX.

        % More comments.
    \end{minted}
\end{frame}

\begin{frame}[fragile]{Commands}
    In \LaTeX{}, \alert{commands} are also called as \alert{macros}.

    \begin{minted}[breaklines, autogobble, linenos] {latex}
        % Optional ----+-----------+
        % Arguments    |           |
        %              v           v
        \documentclass[a4paper, 11pt]{article}
        %                             ^     ^
        %                             |     |
        %                             +-----+--- Arguments
    \end{minted}

    They are just like \alert{functions} in other programming languages.
    You can define \alert{custom} commands / macros on your own (more on that later...).

\end{frame}

\begin{frame}[fragile]{Document Class}

    Every \LaTeX{} document has a \alert{\texttt{\textbackslash documentclass}} declaration at the top.

    \begin{minted}[breaklines, autogobble, linenos] {latex}
        \documentclass[a4paper, 11pt]{article}
        %                             ^     ^
        %                             |     |
        %                             +-----+-- Document Class
    \end{minted}

    There are several types:

    {\begin{minipage}[t] {0.4\textwidth}
        \begin{itemize}
            \item article
            \item book
            \item report
        \end{itemize}
    \end{minipage}
    \hfill
    \begin{minipage}[t] {0.4\textwidth}
        \begin{itemize}
            \item memoir
            \item beamer
        \end{itemize}
    \end{minipage}}

\end{frame}

\begin{frame}[fragile]{Importing Packages}

    You can import various packages using \alert{\textbackslash usepackage\{\}} in \LaTeX{} like so:


    \begin{minted}[breaklines, autogobble, linenos] {latex}
        \documentclass[a4paper, 11pt]{article}

        \usepackage[margin = 0.4in]{geometry}

        \begin{document}

        % Document content goes here.

        \end{document}
    \end{minted}
\end{frame}

\begin{frame}[fragile]{Environments}

    Environments in \LaTeX{} starts with \alert{\textbackslash begin\{\}} and ends with \alert{\textbackslash end\{\}}.

    \begin{minted}[breaklines, autogobble, linenos] {latex}
        %                 +--------+-------------+
        %                 |        |             |  tabularx
        %                 v        v             |  environment
        \begin{tabularx} {\textwidth} {ccc} %  |
        %                               ^  ^     |
        %                               |  |     |
        %                               +--+-----+--- More Arguments
        % Content specific to the
        % environment goes inside here.
        \end{tabularx}
    \end{minted}
\end{frame}

\begin{frame}[fragile]{Another Environment Example}
    \begin{minted}[breaklines, autogobble, linenos] {latex}
        %                                           figure
        \begin{figure} [h]  %                      environment
        %               ^ ^
        %               | |
        %               +-+-------------+--- Optional arguments

        % Content specific to the
        % environment goes inside here.

        \end{figure}
    \end{minted}

    % \alertNote{We'll see these environments later.}

\end{frame}

\begin{frame}[fragile]{Chapters, sections and more...}
    \begin{minted}[breaklines, autogobble, linenos] {latex}
        \chapter{Name of the chapter}

        \section{Name of the section}

        \subsection{Name of the sub-section}

        \subsubsection{Name of the sub-sub-section}

        \paragraph{Start of the paragraph}
    \end{minted}
\end{frame}

% \begin{frame}[fragile]{<++>}
%     \begin{minted}[breaklines, autogobble, linenos] {latex}
%     \end{minted}
% \end{frame}

\end{document}
